%%%%%%%%%%%%%%%%%%%%%%%%%%%%%%%%%%%%%%%%%%%%%%%%%%%%%%%%%%%%%%%%%%%%%%%%%%%
\section{Discussion, conclusion}
\label{SectionDiscussion}
%%%%%%%%%%%%%%%%%%%%%%%%%%%%%%%%%%%%%%%%%%%%%%%%%%%%%%%%%%%%%%%%%%%%%%%%%%%
Dukowicz's acoustic, gravity and surface wave Lagrangian model based on two dispersion relations \cite{dukowicz_2013} has been re-derived with in a fully Eulerian context. Not that this later approach is not physically more coherent but its derivation is just simpler. Acoustic and internal wave rays propagating in an unbounded ocean have first been re-visited with a single dispersion equation. Smith's acoustic modes (\cite{smith_2015}) have been recovered with this model and a both a short and a long-wave approximation of these acoustic modes has been proposed. Well-known internal modes have also been revisited in a compressible, stratified, free-surface ocean. Surface waves (edge waves) have been systematically investigated in a compressible and stratified ocean. Long surface waves have also been revisited questioning their classification as barotropic modes. Dukowicz effort to give a coherent and complete framework for the description of geophysical waves is thus carried on further integrating new wave solutions and clarifying the description of key regions of phase-space such as long waves.\\ 

$(\delta_x,\ \delta_z,\ \omega)$ phase-space has been explored geometrically to identify possible ocean waves and modes: intersections of the inner and boundary dispersion surfaces are localized numerically and are then used as references to derive wave approximations. This original investigation in phase-space when associated with systematic Taylor developments with respect to small compressibility and stratification parameters, provides an adapted approach to circumvent the non-linearity and the transcendental character of the boundary dispersion relation and the (high) fourth-order dependency in the pulsation $\omega$ of the inner dispersion relation. In $(\delta_x,\ \delta_z,\ \omega)$ phase-space, the inner dispersion surface has been decomposed into three distinct branches with a one-to-one correspondence along the $\omega$ axis. For large pulsations, the \textit{acoustic branch} is well described by the simple factorizing function $\omega_a$ and is bounded for vanishing $(\delta_x,\ \delta_z)$ by the acoustic cut-off pulsation $(\omega_{MAM,-})$. Acoustic waves propagating in the ocean as in an unbounded medium (MAW) belongs to this branch together with acoustic modes modified by gravity (MAM) which can be found at the intersection of this acoustic branch with the dispersion boundary surface given by the transcendental relation \ref{EqFullDisperb}.\\
For low pulsations (long waves), the \textit{internal gravity branch} of the inner dispersion surface is in turn well approximated by the internal factorizing function $(\omega_i)$. The pulsation of the internal waves belonging to this branch is bounded by the internal parameter $\epsilon_i$ (the cut-off pulsation is $N$, the Brunt-Väisälä). Internal gravity rays belong to this surface. At the intersection of the lower-pulsation gravity branch with the \textit{boundary dispersion surface}, are found internal gravity modes (MIM).\\
Compressibility-induced perturbations to MIM and gravity-induced perturbations to MAM have been shown to be high-ordered perturbations in small parameters $\epsilon_i$ and $\epsilon_a$. These wave-modes have been shown to be well-separated graphically and analytically. They are well consequently well approximated respectively by the stratification and acoustic factorizing functions $\omega_i$ and $\omega_a$. The situation is somehow different for MSW. This latter wave solution is a linear combination of the real roots $(\omega_\pm)$, but these two roots might not be well-separated for waves with purely imaginary vertical wavenumbers. A consequence is that unlike for real vertical wavenumbers, the contributions of the acoustic factorizing function $\omega_a$ and of the stratification factorizing function $\omega_i$ cannot be meaningfully separated.\\
Well-know relations for internal wave-rays or acoustic waves in an unbounded ocean and for internal-gravity or acoustic wave-modes in a bounded ocean have been recovered. Lower-order perturbations due to stratification and gravity ($\epsilon_i$) and to compressibility ($\epsilon_a$) have been proposed for each type of waves. Acoustic Lamb-waves have been examined as a particular case: they are solution of the proposed (first-order) wave-model only under an additional "rigid-lid" assumption.\\

Between these upper and lower branches of the inner dispersion surface, waves can propagate with middle-range pulsations only if their vertical wavenumber is a purely imaginary complex. The intersection with the \textit{boundary dispersion surface} hosts surface waves (MSW) which can be "modified" by compressibility and by gravity (stratification). The medium-range branch is indeed "folded" both by the ocean compressibility and by the vertical density stratification. In both cases the surface is bounded: an upper bound for vanishing wavenumbers due to the acoustic cut-off an a lower bound for large pulsations due to the Brunt-Väisälä pulsation cut-off. When pulsation is increased from very low pulsations, MSW intersections can change from surface waves modified by stratification to surface waves modified by compressibility. In the transition region, the two wave solutions merge into a single "neutral" solution at the limit of the region where pulsation are purely imaginary complex numbers.
%%Three branches of surface solutions have been identified: one for pure-imaginary vertical wave-numbers $(\delta_z)$ includes MSW solutions, the remaining two (this time for real $\delta_z$) includes MAW, MIW, MAM and MIM. The latter two branches have been shown to be well-separated in $(\delta_x,\ \delta_z,\ \omega)$ space. This confirms, if necessary, that surface, acoustic and internal solutions belong to different branches of dispersion surfaces. They are based on different propagating mechanisms. \\
%%These wave solutions have been first identified graphically in the $(\delta_x,\ \delta_z,\ \omega)$ phase-space. The large-pulsation branch is
In the long-wave limit now, a modified "n=0" gravity mode (MIM-0) exists only in a stratified ocean. It can be associated to a low-frequency oscillation of the stratified water layer. It is part of the small-pulsation branch of the real-$\delta_z$ inner dispersion surface. It thus cannot be the asymptotic limit of the shorter swell-like MSW branch. Usual approximations for long surface waves ($\delta_z=\delta_x$ and $\omega=\delta_x$) can be recovered from the results of the present studies in two ways. It can either be introduced as a long-wave approximation of MSW for homogeneous (non-stratified) ocean or as a long-wave approximation of mode-0 MIM. In this latter case, a vanishing vertical wave-number is recovered only for $\epsilon_i=\epsilon_a=0$.
%For small horizontal wave-number, the wave solution can be viewed as a (long) \textit{barotropic mode} $(n=0)$ since it is on the same inner dispersion branch as MIM which are solutions for $n>0$.
MSW are thus primarily edge wave: the free-surface anomaly is one way or the other translated into a pressure anomaly by gravity, wave propagation is then achieved by a simple compensation mechanism based on the conservation of mass and momentum. If the ocean is homogeneous and incompressible, the pressure is an harmonic function and the horizontal and vertical length-scales are equal $(\delta_x=\delta_z)$. Compressibility and stratification modifies this and as shown in Section \ref{subsubsectioniR}, in this case, the difference between the horizontal and vertical wave-number increases. The vertical wave-number decreases consequently faster and the MSW variations become very small over the vertical. If $\delta_x$ is further decreased, $\delta_z$ finally vanishes for small but finite $\delta_x$.\\
If the horizontal wave-number keeps on decreasing, a surface wave can only propagate as a mode-0 MIM. In this case, the vertical wave-number must increase for a decreasing horizontal wave-number due to the stratification barrier: the stronger the stratification, the shallower the in-depth penetration of the surface wave.\\

In a stratified ocean (whether compressible or not), long surface waves (LMSW in the vicinity of the origin in phase-space) cannot propagate and are replaced by (mode-0) MIM solutions. The longest surface wave that can propagate is barotropic (depth-independent): its horizontal wave-number is $\delta_{x,lmsw}$ and its vertical wave-number is zero. When the stratification weakens, $\delta_{x,lmsw}$ tends toward 0 and the resulting long waves approaches LSW (with $\sqrt{g H}$ phase and group velocity and $\delta_x=\delta_z$).
Propagation in an homogeneous ocean can be studied with the present model by setting to zero the stratification parameter $\epsilon_i$ together with the last (advective) term on the right-hand-side of the inner dispersion relation \ref{EqFullDispera}: $-(\epsilon_i^2+\epsilon_a^2)^2/4$. 

Further inspection of MSW waves has shown that there exists a particular triplet of properties $(\delta_x,\ \delta_z,\ \omega)$ for which the pair of real roots \ref{solseq} merges and the discriminant of the second-order polynomial equation in the pulsation vanishes. This double root is located in the region where the inner dispersion surface is vertical and contributions due to compressibility and stratification are smaller. This MSW solution is very peculiar in the sense that it is located at the edge of the region of $(\delta_x,\ \delta_z,\ \omega)$ phase-space where wave solutions are divergent as time goes on. Ocean waves originating in this region of phase space might have singular behaviour.\\