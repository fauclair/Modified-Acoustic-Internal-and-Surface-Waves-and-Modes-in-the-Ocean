%%%%%%%%%%%%%%%%%%%%%%%%%%%%%%%%%%%%%%%%%%%%%%%%%%%%%%%%%%%%%%%%%%%%%%%%%%%
\section{Conclusion}
\label{SectionDiscussion}
%%%%%%%%%%%%%%%%%%%%%%%%%%%%%%%%%%%%%%%%%%%%%%%%%%%%%%%%%%%%%%%%%%%%%%%%%%%
Several authors have made comprehensive studies on waves in the ocean. The primary purpose of this paper was to extend these studies by going at an higher order in the study of dispersion relations. This allows to look at the combined effect of compressibility and stratification on acoustic, internal and free surface waves. \\
We first proved in Appendix \ref{kzreal} that, for physically sounded values of the compressibility $\epsilon_a$ and stratification $\epsilon_i$ parameters, the solutions are either vertically propagating ($\delta_z$ real) or vertical evanescent $\delta_z$ pure imaginary.
The acoustic and gravity waves, satisfying the inner dispersion relation, have then been described in an unbounded ocean. The addition of the surface dispersion relation leads to the characterization of internal modes (MIM) and acoustic modes (MAM) along with surface waves (MSW). The analysis has been first performed in light of a graphical study of the inner and boundary dispersion relations and their intersections. This has allowed us to characterized triplets ($\delta_x, \delta_z, \omega$) of solutions. Then Taylor developments have been performed at an order higher than the usual dispersion relations where combined effects of compressibility, stratification and freed surface boundary conditions are not taken into account.\\
The main results have been summarized qualitatively and quantitatively in \ref{summarybounded} and corresponding analytic solutions in Appendix \ref{solutionana}.\\
These developments are particularly useful for the validation of recent free surface non-hydrostatic and potentially compressible numerical ocean models.
%At high frequency, the {\it acoustic branch} is well described by the simple 
%
%The Lagrangian model of \cite{dukowicz_2013} for acoustic, gravity and surface waves, based on two dispersion relations, is revisited in the present study within a fully Eulerian approach. The latter is not physically more coherent but its derivation and analysis are simpler. Acoustic and internal wave rays propagating in an unbounded ocean are first evaluated with a single dispersion equation. Smith's acoustic modes \citep{smith_2015} are recovered and both short and long-wave approximations are proposed. Then, well-known internal modes and surface waves (edge waves) are revisited in a compressible, stratified, free-surface ocean. Long surface waves are analyzed and their classification as barotropic modes questioned. This work complements efforts made by Dukowicz to provide a coherent and complete framework for the description of geophysical waves, integrating new wave solutions and clarifying the description of key phase-space regions such as long waves.
%
%A graphical analysis is proposed, using the $(\delta_x,\ \delta_z,\ \omega)$ phase-space to classify ocean waves and modes: intersections of the inner and boundary dispersion surfaces are localized numerically and used as references to derive wave approximations. This original investigation in phase-space, associated with Taylor developments with respect to small compressibility and stratification parameters, provides an adapted approach to circumvent the nonlinear and transcendental characters of the boundary dispersion relation and the (high) fourth-order dependency in frequency $\omega$ of the inner dispersion relation. In $(\delta_x,\ \delta_z,\ \omega)$ phase-space, the inner dispersion surface is decomposed into three distinct branches with a one-to-one correspondence along the $\omega$ axis. At high frequency, the \textit{acoustic branch} is well described by the simple factorizing function $\omega_a$ and is bounded for vanishing $(\delta_x,\ \delta_z)$ by the acoustic cut-off frequency $(\omega_{MAM,-})$. Acoustic waves propagating in the ocean as in an unbounded medium (MAW) belongs to this branch together with acoustic modes modified by gravity (MAM) which can be found at the intersection of this acoustic branch with the dispersion boundary surface given by the transcendental relation \ref{EqFullDisperb}.
%
%At low frequency (long waves), the \textit{internal gravity branch} of the inner dispersion surface is in turn well approximated by the internal factorizing function $(\omega_i)$. The frequency of internal waves in this branch is bounded by the internal parameter $\epsilon_i$ (the cut-off frequency is $N$, the Brunt-Väisälä frequency). Internal gravity rays belong to this surface. At the intersection of the lower-frequency gravity branch with the \textit{boundary dispersion surface} are found internal gravity modes (MIM).
%
%Compressibility perturbations to MIM and gravity perturbations to MAM are shown to be high-order perturbations in small parameters $\epsilon_i$ and $\epsilon_a$. These wave-modes are well-separated graphically and analytically. They are thus well approximated respectively by the stratification and acoustic factorizing functions $\omega_i$ and $\omega_a$. The situation is somehow different for MSW. The wave solution in this case is a linear combination of real roots $(\omega_\pm)$, but these two roots might not be well-separated for waves with purely imaginary vertical wavenumbers. A consequence is that unlike for real vertical wavenumbers, the contributions of acoustic and stratification factorizing functions $\omega_a$ and $\omega_i$ cannot be meaningfully separated.
%
%Well-know relations for internal wave-rays or acoustic waves in an unbounded ocean and for internal-gravity or acoustic wave-modes in a bounded ocean are recovered. Lower-order perturbations due to stratification and gravity ($\epsilon_i$) and to compressibility ($\epsilon_a$) are proposed for each type of waves. Acoustic Lamb-waves are examined as a particular case: they are solutions of the proposed (first-order) wave-model only under an additional "rigid-lid" assumption.\\
%
%Between these upper and lower branches of the inner dispersion surface, waves can propagate with middle-range frequencies only if their vertical wavenumber is a purely imaginary complex. The intersection with the \textit{boundary dispersion surface} hosts surface waves (MSW) which can be "modified" by compressibility and stratification. The medium-range branch is indeed "folded" by both compressibility and stratification. In both cases, the surface is bounded: an upper bound for vanishing wavenumbers due to acoustic cut-off; and a lower bound for high frequencies due to Brunt-Väisälä cut-off. When frequency is increased from very low values, MSW intersections can change from surface waves modified by stratification to surface waves modified by compressibility. In the transition region, the two wave solutions merge into a single "neutral" solution at the limit where frequency is purely imaginary.
%%%Three branches of surface solutions have been identified: one for pure-imaginary vertical wave-numbers $(\delta_z)$ includes MSW solutions, the remaining two (this time for real $\delta_z$) includes MAW, MIW, MAM and MIM. The latter two branches have been shown to be well-separated in $(\delta_x,\ \delta_z,\ \omega)$ space. This confirms, if necessary, that surface, acoustic and internal solutions belong to different branches of dispersion surfaces. They are based on different propagating mechanisms. \\
%%%These wave solutions have been first identified graphically in the $(\delta_x,\ \delta_z,\ \omega)$ phase-space. The large-pulsation branch is
%
%In the long-wave limit, a modified "n=0" gravity mode (MIM-0) exists only in a stratified ocean. It can be associated with a low-frequency branch of the real-$\delta_z$ inner dispersion surface. Therefore, it cannot be the asymptotic limit of shorter swell-like MSW branch. Usual approximations for long surface waves ($\delta_z=\delta_x$ and $\omega=\delta_x$) can be recovered from our results in two ways. It can either be introduced as a long-wave approximation of MSW for an homogeneous (non-stratified) ocean or as a long-wave approximation of mode-0 MIM. In the latter case, a vanishing vertical wavenumber is recovered only for $\epsilon_i=\epsilon_a=0$.
%%For small horizontal wave-number, the wave solution can be viewed as a (long) \textit{barotropic mode} $(n=0)$ since it is on the same inner dispersion branch as MIM which are solutions for $n>0$.
%
%MSW are thus primarily edge waves: the free-surface anomaly is one way or the other translated into a pressure anomaly by gravity; wave propagation is then achieved by a simple compensation mechanism based on the conservation of mass and momentum. If the ocean is homogeneous and incompressible, the pressure is a harmonic function and horizontal and vertical length-scales are equal $(\delta_x=\delta_z)$. With compressibility and stratification, horizontal and vertical wavenumbers are different (Sec. \ref{subsubsectioniR}). If $\delta_x$ decreases, $\delta_z$ decreases even faster and vertical MSW variations become very small. If $\delta_x$ further decreases, $\delta_z$ finally vanishes for small but finite $\delta_x$. If the horizontal wavenumber keeps on decreasing, a surface wave can only propagate as a mode-0 MIM. In this case, the vertical wavenumber must increase again due to a stratification barrier: the stronger the stratification, the shallower the penetration depth of surface waves.
%
%In a stratified ocean (whether compressible or not), long surface waves (LMSW in the vicinity of the origin in phase-space) cannot propagate and are replaced by (mode-0) MIM solutions. The longest surface wave that can propagate is barotropic (depth-independent): its horizontal wavenumber is $\delta_{x,lmsw}$ and its vertical wavenumber is zero. When stratification weakens, $\delta_{x,lmsw}$ tends to 0 and the resulting long wave approaches LSW (with $\sqrt{g H}$ phase and group velocity and $\delta_x=\delta_z$).
%Propagation in an homogeneous ocean can be studied with the present model by setting to zero the stratification parameter $\epsilon_i$, together with the last (advective) term on the right-hand-side of inner dispersion relation \ref{EqFullDispera}: $-(\epsilon_i^2+\epsilon_a^2)^2/4$. 
%
%Further inspection of MSW waves shows that there exists a particular triplet of properties $(\delta_x,\ \delta_z,\ \omega)$ for which the pair of real roots \ref{solseq} merges and the discriminant of the second-order polynomial equation in frequency vanishes. This double root is located in the region where the inner dispersion surface is vertical and contributions due to compressibility and stratification are small. This MSW solution is very peculiar in the sense that it is located at the edge of the region of $(\delta_x,\ \delta_z,\ \omega)$ phase-space where wave solutions can diverge. Ocean waves originating in this region of phase space may thus have a singular behaviour.\\