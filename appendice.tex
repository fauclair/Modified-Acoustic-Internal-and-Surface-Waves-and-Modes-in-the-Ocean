\appendix
\section{$\delta_z$ real or purely imaginary}
\label{kzreal}
In this section, we prove that under the condition of smallness of parameters $\epsilon_i$ and $\epsilon_a$, $\delta_z$ is either real or pure imaginary and thus that the frequency $\omega$ is real.\\
The eigenvalue problem \ref{EqDimF} and the surface boundary condition \ref{EqDimFb} are rewritten in non-dimensional form as:
\begin{align}
G''(s)
+
\delta_z^2
G(s)&=0 \label{sturm1}\\
G(1)&=1 \label{sturm2}\\
G(0)&=0 \label{sturm3}
\end{align}
\begin{equation}
G'(1)+\left(
\frac{\epsilon_i^2+\epsilon_a^2}{2}-\frac{\delta_x^2}{\omega^2}
\right) G(1)=0
\label{eqnormal}
\end{equation}
where $\displaystyle s=\frac{z+H}{H}$, $G(s(z))=F(z)$ and $\displaystyle \epsilon_i^2=\frac{N^2H}{g}, \epsilon_a^2=\frac{gH}{c_s^2},\delta_x=k_xH,\delta_z=k_zH,\omega=\Omega\sqrt{\frac{H}{g}}$.\\
$\delta_z$ is linked to $\delta_x$ and $\omega$ by the inner dispersion relation
\begin{equation}
\delta_z^2=\left(\delta_x^2\frac{\epsilon_i^2-\omega^2}{\omega^2}
+\epsilon_a^2\omega^2-\frac{(\epsilon_i^2+\epsilon_a^2)^2}{4}\right)
\label{innerannex}
\end{equation}
Multiplying \ref{sturm1} by $\overline{G(s)}$ and integrating over $[0,1]$ we get:
\[
\int_0^1G''(s)\overline{G(s)}{\rm d}s+\delta_z^2\int_0^1|G(s)|^2{\rm d}s=0
\]
Integration by parts leads to:
\[
-\int_0^1|G'(s)|^2{\rm d}s+\delta_z^2\int_0^1|G(s)|^2{\rm d}s+G'(1)\overline{G(1)}-G'(0)\overline{G(0)}=0
\]
and using \ref{sturm2}, \ref{sturm3}, \ref{eqnormal}
\[
\delta_z^2 \int_0^1|G(s)|^2{\rm d}s+\left(\frac{\delta_x^2}{\omega^2}-
\frac{\epsilon_i^2+\epsilon_a^2}{2}
\right)=\int_0^1|G'(s)|^2{\rm d}s
\]
Using the Poincar\'e inequality $\displaystyle \int_0^1|G(s)|^2{\rm d}s\le \int_{0}^1|G'(s)|^2{\rm d}s$ and $\displaystyle 1=|G(1)|^2 \le \int_{0}^1|G'(s)|^2 {\rm d}s$, we obtain:
\begin{equation}
\delta_z^2 \mu + \left(
\frac{\delta_x^2}{\omega^2}-\frac{\epsilon_i^2+\epsilon_a^2}{2}
\right)
\nu = 1
\label{sturm4}
\end{equation}
with $0\le \mu \le 1$ and $0\le \nu \le 1$.\\
Taking the imaginary part of \ref{sturm4} and using \ref{innerannex}
\[
\mu \left( \epsilon_a^2 \Im[\omega^2] +\delta_x^2 \epsilon_i^2 \Im[1/\omega^2]\right)+\nu \delta_x^2\Im[1/\omega^2]=0
\]
or, using $\Im[1/\omega^2]=-\Im[\omega^2]/|\omega|^4$,
\begin{equation}
\Im[\omega^2]\left(\mu \left(\epsilon_a^2 -\epsilon_i^2 \frac{\delta_x^2}{|\omega|^4} \right) -\nu \frac{\delta_x^2}{|\omega|^4} \right)=0
\label{constraint1}
\end{equation}
Let us assume that $\omega$ is neither real or pure imaginary. This implies $\Im[\omega^2] \ne 0$ and \ref{constraint1} leads to:
\begin{equation}
\mu\left(\epsilon_a^2\frac{|\omega|^4}{\delta_x^2} -\epsilon_i^2\right)-\nu=0
\label{eqmunu1}
\end{equation}
We will show below that, in this case, solutions can exist only for non physical values of $\epsilon_i, \epsilon_a$ satisfying $\max(\epsilon_i,\epsilon_a) > \sqrt{2}$.\\
Taking the real part of \ref{sturm4}
\[
\mu \left(-\delta_x^2+\delta_x^2\epsilon_i^2\Re[1/\omega^2]-\frac{(\epsilon_i^2+\epsilon_a^2)^2}{4}+\epsilon_a^2 \Re[\omega^2]\right)
+\nu \left(
\delta_x^2 \Re[\frac{1}{\omega^2}]-\frac{\epsilon_i^2+\epsilon_a^2}{2}
\right)=1
\]
or, using $\Re[1/\omega^2]=\Re[\omega^2]/|\omega|^4$,
\[
\mu \left(-\delta_x^2-\frac{(\epsilon_i^2+\epsilon_a^2)^2}{4}+(\frac{\delta_x^2\epsilon_i^2}{|\omega|^4}+\epsilon_a^2 )\Re[\omega^2]\right)
+\nu \left(
\frac{\delta_x^2}{|\omega|^4} \Re[\omega^2]-\frac{\epsilon_i^2+\epsilon_a^2}{2}
\right)=1.
\]
\ref{eqmunu1} allows to simplify in:
\begin{equation}
\mu \left(-\delta_x^2-\frac{(\epsilon_i^2+\epsilon_a^2)^2}{4}+2\epsilon_a^2\Re[\omega^2] \right)
-\nu \left(
\frac{\epsilon_i^2+\epsilon_a^2}{2}
\right)=1
\label{eqmunu2}
\end{equation}
Eqs \ref{eqmunu1} and \ref{eqmunu2} can be summarized in 
\begin{equation}
\begin{array}{l}
\alpha \mu  - \beta \nu = 1\\
\gamma \mu - \nu=0
\end{array}
\label{eqmunu}
\end{equation}
with
\[
\alpha= -\delta_x^2-\frac{(\epsilon_i^2+\epsilon_a^2)^2}{4}+2\epsilon_a^2\Re[\omega^2], \beta = \frac{\epsilon_i^2+\epsilon_a^2}{2}, \gamma =  \epsilon_a^2\frac{|\omega|^4}{\delta_x^2} -\epsilon_i^2
\]
Using $\beta > 0$, it is easy to check that \ref{eqmunu} has solutions $(\mu,\nu)$ positive and with magnitude less than one only if
\[
\gamma>1 \mbox{ and }\alpha > (1 + \beta ) \gamma
\]
or
\[
0\le \gamma\le 1 \mbox{ and }\alpha \ge 1 + \beta \gamma
\]
The conditions above immediately exclude the cases $\epsilon_a=0$ (which leads to $\gamma < 0$ ) and $\Re[\omega^2] \le 0$ (which leads to $\alpha <  0$ ). They will not be considered below.
\begin{itemize}
	\item First case: $0\le \gamma \le 1$\\
	This implies:
	\begin{equation}
	|\omega|^4 \le \frac{\delta_x^2}{\epsilon_a^2}(1+\epsilon_i^2)
	\label{maj1}
	\end{equation}
	$\alpha \ge 1 + \beta \gamma$ writes:
	\[
	\Re[\omega^2] \ge  \frac{1}{2\epsilon_a^2}\left[
	1+\delta_x^2+\frac{(\epsilon_a^2+\epsilon_i^2)^2}{4}
	+\frac{\epsilon_i^2+\epsilon_a^2}{2}\left(
	\epsilon_a^2\frac{|\omega|^4}{\delta_x^2} -\epsilon_i^2
	\right)
	\right]
	\]
	Now using the inequalities $|\Re[\omega^2]|^2 \le |\omega|^4$ and \ref{maj1}, we get
	\begin{equation}
	\frac{\delta_x^2}{\epsilon_a^2}(1+\epsilon_i^2)
	\ge
	|\omega|^4
	\ge 
	\left(
	\frac{1}{2\epsilon_a^2}\left[
	1+\delta_x^2+\frac{(\epsilon_a^2+\epsilon_i^2)^2}{4}
	+\frac{\epsilon_i^2+\epsilon_a^2}{2}\left(
	\epsilon_a^2\frac{|\omega|^4}{\delta_x^2} -\epsilon_i^2
	\right)
	\right]
	\right)^2
	\label{systeme1}
	\end{equation}
	Using a computing algebra software to simplify the technical exercice, we can prove that \ref{systeme1} has solutions if and only if
	\[
	\epsilon_i^2 > \sqrt{4+\epsilon_a^4}
	\]
	which requires $\epsilon_i > \sqrt{2}$.
	\item Second case: $\gamma > 1$\\
	This implies:
	\begin{equation}
	|\omega|^4 > \frac{\delta_x^2}{\epsilon_a^2}(1+\epsilon_i^2)
	\label{maj2}
	\end{equation}
	$\alpha > (1 + \beta ) \gamma$ can be written as:
	\[
	\gamma < \frac{\alpha}{1+\beta}
	\]
	or
	\[
	|\omega|^4 < \frac{\delta_x^2}{\epsilon_a^2}\left[
	\epsilon_i^2
	+\frac{1}{1+\frac{\epsilon_a^2+\epsilon_i^2}{2}}
	\left(
	-\delta_x^2-\frac{(\epsilon_i^2+\epsilon_a^2)^2}{4}+2\epsilon_a^2\Re[\omega^2]
	\right)
	\right]
	\]
	Now using $0 \le \Re[\omega^2] \le |\omega|^2$ and adding \ref{maj2}, we get:
	\[
	\frac{\delta_x^2}{\epsilon_a^2}(1+\epsilon_i^2)
	<
	|\omega|^4 < \frac{\delta_x^2}{\epsilon_a^2}\left[
	\epsilon_i^2
	+\frac{1}{1+\frac{\epsilon_a^2+\epsilon_i^2}{2}}
	\left(
	-\delta_x^2-\frac{(\epsilon_i^2+\epsilon_a^2)^2}{4}+2\epsilon_a^2|\omega|^2
	\right)
	\right]
	\]
	which has non trivial solutions if and only if
	\[
	\epsilon_a^2 > 2+ \epsilon_i^2
	\]
	which requires $\epsilon_a > \sqrt{2}$.
\end{itemize}
This concludes the proof. If $\epsilon_a=0$ or $\max(\epsilon_i,\epsilon_a)\le \sqrt{2}$, then $\Im[\omega^2]$ is zero. This also leads to $\Im[\delta_z^2]=0$ and we conclude that, under these conditions, $\delta_z$ is either real or purely imaginary.
